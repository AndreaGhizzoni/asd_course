\documentclass[12pt]{article} 
\usepackage[margin=1in]{geometry}
\usepackage{array}
\usepackage{tabularx} % for tables
\usepackage{amsmath}
\usepackage{mathtools}
\usepackage{enumerate}
\usepackage{enumitem} %% for custom list
%\usepackage{graphicx} %% for images
%\usepackage{multirow} %% for tables
%\usepackage{bigints}  %% for integrals
%\usepackage[T1]{fontenc} %% for pipe simbol
%\usepackage{wrapfig} %% used to wrap img around text: https://goo.gl/SrlHQI
%\usepackage{bytefield} %% for drawing packet structure
%\usepackage[dvipsnames]{xcolor} %% to use with VerbatimInput to import stuff from text file
%\usepackage{fancyvrb}

% used for tabbed spacing
%\newcommand{\itab}[1]{\hspace{0em}\rlap{#1}}
%\newcommand{\tabtwo}[1]{\hspace{.2\textwidth}\rlap{#1}}
%\newcommand{\tabthree}[1]{\hspace{.3\textwidth}\rlap{#1}}
%\newcommand{\tabfour}[1]{\hspace{.4\textwidth}\rlap{#1}}

% using \code{come code monospaced}
%\def\code#1{\texttt{#1}}

% redefine \VerbatimInput
%\RecustomVerbatimCommand{\VerbatimInput}{VerbatimInput}%
%{fontsize=\footnotesize,
 %
% frame=lines,  % top and bottom rule only
% framesep=2em, % separation between frame and text
% rulecolor=\color{Gray},
 %
 %label=\fbox{\color{Black}data.txt},
 %labelposition=topline,
 %
% commandchars=\|\(\), % escape character and argument delimiters for commands within the verbatim
% commentchar=*        % comment character
%}

\renewcommand{\arraystretch}{1.4}

\delimitershortfall-1sp
\newcommand\abs[1]{\left|#1\right|}


\begin{document}
\date{}
%\author{Andrea Ghizzoni \\
%some other info}
%\title{\vspace{-11ex}} %% used for no title
%\maketitle

%\section{}\label{}
%\subsection{}\label{}
%\subsubsection{}\label{}
%\paragraph{•}

\subsection*{Sort}
\begin{tabularx}{\textwidth}{ p{6cm} p{2.7cm} p{1.3cm} | X }
    \hline
    Insertion Sort & $O(n^2)$ & $O(n)$ &  \\ 
    \hline
	Counting Sort  & $O(k+n)$ & $O(n)$ &  {\footnotesize conoscendo il range di valori} \\ 
	\hline
	Heap Sort      & $O(n\log(n))$ &  ------ &  \\ 
	\hline
	Quick Sort     & $O(n\log(n))$ & $O(n^2)$  & {\footnotesize nel caso di vettore gi\'a ordinato} \\ 
	\hline
\end{tabularx}
\\
\subsection*{Grafo}
\begin{tabularx}{\textwidth}{ p{6cm} p{2.7cm} p{1.3cm} | X }
    \hline
    BFS - DFS                   & $O(n+m)$ & ------ &  \\ \hline
	Componenti Connesse         & $O(n+m)$ & ------ &  \\ \hline
	Strong Componenti Connesse  & $O(n+m)$ & ------ &  \\ \hline
	Ciclo                       & $O(n+m)$ & ------ &  \\ \hline
\end{tabularx}
\\
\subsection*{Cammini Minimi}
\begin{tabularx}{\textwidth}{ p{6cm} p{2.7cm} | X }
    \hline
    	Dijkstra {\scriptsize (coda a priorit\'a)}  & 
    	$O(n^2)$ & 
    	{\footnotesize deleteMin() $\forall$ nodo $\Rightarrow O(n)n $} \\ 
    \hline
		Johnson  {\scriptsize (coda a priorit\'a con heap binario)}  & 
		$O(m\log(n))$ & 
		{\footnotesize migliore per grafi sparsi, m = num archi} \\ 
	\hline
		Belman-Ford-Moore {\scriptsize (coda)} & 
		$O(n+m)$ &
		{\footnotesize il migliore perch\'e lavora con archi negativi} \\ 
	\hline
\end{tabularx}
\\
\subsection*{Cammini Minimi fra tutte le Coppie di Nodi}
\begin{tabularx}{\textwidth}{ p{6cm} p{2.7cm} | X }
    \hline
    	Floyd-Warshall & 
    	$O(n^3)$ & 
    	\\ 
    \hline
\end{tabularx}
\\
\subsection*{Albero di Copertura Minima}
\begin{tabularx}{\textwidth}{ p{6cm} p{2.7cm} | X }
    \hline
    	Kruskal & 
    	$O(m\log(n))$ & 
    	{\footnotesize ordina archi in modo crescente di peso e utilizza MFSET} \\ 
    \hline
   		Prim & 
    	$O(m\log(n))$ & 
    	{\footnotesize aggiunge sempre l'arco con peso minore partendo da un nodo radice} \\ 
    \hline
\end{tabularx}
\\
\subsection*{Rete di Flusso}
\begin{tabularx}{\textwidth}{ p{6cm} p{2.7cm} | X }
    \hline
    	Ford-Fulkerson & 
    	$O( \abs{f^{\star}}(m+n) )$ & 
    	{\footnotesize cammino aumentante con DFS che csta $O(n+m)$} \\ 
    \hline
   		Edmond-Kapp & 
    	$O(nm^2)$ & 
    	{\footnotesize cammini aumentanti con BFS} \\ 
    \hline
\end{tabularx}

\newpage

\subsection*{Master Theorem}
\[
 T(n) = \sum_{1 \leq i \leq h}^{} a_i T(n-i) + cn^b
\]
ponendo $\alpha = \sum_{1 \leq i \leq h}^{} a_i$ si ottiene:
\begin{enumerate}[noitemsep]
	\item $T(n)$ \'e $O(n^{b+1})$ se $a = 1$
	\item $T(n)$ \'e $O(a^nn^b)$ se $a \geq 2$
\end{enumerate}


\[
 T(n) = a T(n/b) + cn^{\beta}
\]
ponendo $\alpha = \frac{\log(a)}{\log(b)}$ si ottiene:
\begin{enumerate}[noitemsep]
	\item $T(n)$ \'e $O(n^{\alpha})$ se $\alpha > \beta$
	\item $T(n)$ \'e $O(n^{\alpha}\log(n))$ se $\alpha = \beta$
	\item $T(n)$ \'e $O(n^{\beta})$ se $\alpha < \beta$
\end{enumerate}

\newpage
\subsection*{Lista con Puntatori}
\begin{tabularx}{\textwidth}{ p{6cm} p{2.7cm} | X }
    \hline
    	costruzione & $O(1)$ & \\ 
    \hline
		insert & $O(1)$ & \\
	\hline
		remove & $O(1)$ & \\ 
	\hline
		get & $O(n)$ & \\
	\hline
		exists & $O(n)$ & \\
	\hline
\end{tabularx}

\subsection*{Stack e Queue}
\begin{tabularx}{\textwidth}{ p{6cm} p{2.7cm} | X }
    \hline
    	costruzione & $O(1)$ & \\ 
    \hline
		push, enqueue & $O(1)$ & {\footnotesize solo in testa} \\
	\hline
		pop, top & $O(1)$ & {\footnotesize solo in testa} \\ 
	\hline
		top, dequeue & $O(1)$ & \\
	\hline
		isEmpty & $O(1)$ & \\
	\hline
\end{tabularx}

\subsection*{Tabelle Hash - con ABR}
\begin{tabularx}{\textwidth}{ p{6cm} p{2.7cm} | X }
    \hline
    	costruzione & $O(1)$ & \\ 
    \hline
		insert & $O(\log(n))$ & \\
	\hline
		lookup & $O(\log(n))$ & \\ 
	\hline
		remove & $O(\log(n))$ & \\
	\hline
\end{tabularx}

\subsection*{Alberi Binari di ricerca}
\begin{tabularx}{\textwidth}{ p{6cm} p{2.7cm} | X }
    \hline
    	costruzione & $O(1)$ & \\ 
    \hline
		insertNode & $O(\log(n))$ & \\
	\hline
		lookupNode & $O(\log(n))$ & \\ 
	\hline
		removeNode & $O(\log(n))$ & \\
	\hline
\end{tabularx}

\newpage
\subsection*{Grafo con Matrice di Adiacenza}
\begin{tabularx}{\textwidth}{ p{6cm} p{2.7cm} | X }
    \hline
    	costruzione & $O(1)$ & \\ 
    \hline
		adj & $O(n^2)$ & \\
	\hline
		V & $O(n)$ & \\ 
	\hline
		spazio di memoria & $O(n^2)$ & \\
	\hline
		visita & $O(n+m)$ & \\
	\hline
\end{tabularx}

\subsection*{Grafo con Liste di Adiacenza}
\begin{tabularx}{\textwidth}{ p{6cm} p{2.7cm} | X }
    \hline
    	costruzione & $O(1)$ & \\ 
    \hline
		adj & $O(n)$ & \\
	\hline
		V & $O(n)$ & \\ 
	\hline
		spazio di memoria & $O(n)$ & \\
	\hline
		visita & $O(n+m)$ & \\
	\hline
\end{tabularx}

\subsection*{Merge Find Set - basata su Foresta con compressione dei cammini ed Euristica sul Rango}
\begin{tabularx}{\textwidth}{ p{6cm} p{2.7cm} | X }
    \hline
    	costruzione & $O(n)$ & \\ 
    \hline
    	merge & $O(1)$ & \\
	\hline
		find & $O(1)$ & \\
	\hline
\end{tabularx}

\end{document}